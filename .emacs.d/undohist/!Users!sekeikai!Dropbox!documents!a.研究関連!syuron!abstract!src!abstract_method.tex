
((digest . "d66647fdd8b2608f2124a11945ad4a8a") (undo-list nil (1145 . 1149) nil (#("1" 0 1 (fontified t)) . -1145) (t 20738 . 12881) ((marker . 1145) . -1) nil (#("
" 0 1 (fontified t)) . 3902) ((marker . 3905) . -1) nil (#("般的にはこの二つの状況をそれぞれに可視化した主成分プロット,あるいは二つの図を重ねあわせたバイプロットを通して結果を解釈する.
" 0 64 (fontified t)) . 3903) nil (#("主成分分析の結果は,元の観測値 (対象) に対応した変換後の値である主成分得点と,各々の主成分得点に対する変数の重みに相当する主成分負荷量として得られ,一" 0 16 (fontified t) 16 17 (font-lock-face "rainbow-delimiters-depth-1-face" rear-nonsticky t fontified t) 17 19 (fontified t) 19 20 (font-lock-face "rainbow-delimiters-depth-1-face" rear-nonsticky t fontified t) 20 77 (fontified t)) . 3903) nil (#("
" 0 1 (fontified t)) . 3903) nil (#("これは,データのばらつきを最もよく表す方向,すなわち分散が最大となる方向に主成分を設定することによって実現できる.
" 0 58 (fontified t)) . 3903) nil (#("情報損出量を最小にするということは,得られる情報量を最大にすることの裏返しである.
" 0 42 (fontified t)) . 3903) nil (#("主成分とは総合的指標のことである.
" 0 18 (fontified t)) . 3903) nil (#("準化された行列) の特異値分解によって得ることができる.
" 0 7 (fontified t) 7 8 (font-lock-face "rainbow-delimiters-unmatched-face" rear-nonsticky t fontified t) 8 29 (fontified t)) . 3903) nil (#("主成分は,分散共分散行列 (あるいは相関係数行列) に対する固有値分解あるいは,分散共分散行列 (相関係数行列) に対応した偏差行列 (相関係数行列の場合には標" 0 13 (fontified t) 13 14 (font-lock-face "rainbow-delimiters-depth-1-face" rear-nonsticky t fontified t) 14 24 (fontified t) 24 25 (font-lock-face "rainbow-delimiters-depth-1-face" rear-nonsticky t fontified t) 25 48 (fontified t) 48 49 (font-lock-face "rainbow-delimiters-depth-1-face" rear-nonsticky t fontified t) 49 55 (fontified t) 55 56 (font-lock-face "rainbow-delimiters-depth-1-face" rear-nonsticky t fontified t) 56 67 (fontified t) 67 68 (font-lock-face "rainbow-delimiters-depth-1-face" rear-nonsticky t fontified t) 68 80 (fontified t)) . 3903) (t 20738 . 12855) ((marker . 3905) . -80) nil undo-tree-canary))

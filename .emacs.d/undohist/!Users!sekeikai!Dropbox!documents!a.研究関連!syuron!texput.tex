
((digest . "5d654eba332d8bee3aba8c4b6014fb12") (undo-list (1 . 2478) (#("%#BIBTEX jbibtex -kanji=euc syuron
% -*- coding: euc-jp -*-
% vim:set fe=euc-jp
% 半角カナ、規格外文字を=に変換 (text-adjust-codecheck-buffer)
% 全角英数文字を半角に置換 (progn (text-adjust-hankaku-buffer) t)
% 句読点を「、。」に置換 (text-adjust-kutouten-buffer)
% 全角文字と半角文字の間に空白を入れる (text-adjust-space-buffer)
% 全て実行 (text-adjust-buffer)
%\\documentclass[a4j,12pt]{jbook}
\\documentclass[a4j,10pt]{jarticle}
%\\documentclass[mentuke,letter]{ieej}
\\usepackage[dvips]{graphics}
\\usepackage[dvipdfmx]{graphicx}
\\usepackage{graphicx}
\\pagestyle{headings}
\\usepackage{enumerate}
\\usepackage{algorithm}
\\usepackage{algorithmic}
%\\newcommand{\\AND}{\\algorithmicand{}}
%\\renewcommand{\\baselinestretch}{1.0}
\\renewcommand{\\baselinestretch}{1.0}
% \\setlength{\\topmargin}{20mm}
% \\addtolength{\\topmargin}{-2in}
% \\setlength{\\textwidth}{193mm}
% \\setlength{\\textheight}{278mm}
% \\setlength{\\oddsidemargin}{10mm}
% \\addtolength{\\oddsidemargin}{-1in}
% \\setlength{\\evensidemargin}{10mm}
% \\addtolength{\\evensidemargin}{-1in}
% \\setlength{\\abovecaptionskip}{0mm}
% \\setlength{\\belowcaptionskip}{0mm}
% \\setlength{\\textheight}{240mm}
%\\renewcommand{\\bibname}{参考文献}
%%% Gceneterの定義
\\newcommand{\\Gcenter}[2]{
  \\dimen0=\\ht\\strutbox%	%					%1行の高さに
  \\advance\\dimen0\\dp\\strutbox% %				%一行の深さを加えて
  \\multiply\\dimen0 by#1% 						%行数で乗じて
  \\divide\\dimen0 by2% %						%等分し
  \\advance\\dimen0 by -.5\\normalbaselineskip% 	%半行だけ上げる.
  \\raisebox{-\\dimen0}[0pt][0pt]{#2}} %		%求めた値の分だけ字下げする
  % 定義
\\newcommand{\\argmax}{\\mathop{\\rm arg~max}\\limits}
\\newcommand{\\argmin}{\\mathop{\\rm arg~min}\\limits}

\\begin{document}

\\begin{document}
\\begin{titlepage}
 \\begin{center}
  \\vspace*{80pt}
  {\\huge
  修士論文                                             \\\\
  オプティカルフローを用いた発話態度の認識に関する研究 \\\\
  \\vspace*{160pt}
  指導教員  木村春彦 教授                              \\\\
  \\vspace*{80pt}
  平成 25 年 1 月 xx 日 提出                          \\\\
  金沢大学 大学院 自然科学研究科                       \\\\
  博士前期課程 電子情報工学専攻                        \\\\
  人工知能研究室                                       \\\\
  脊渓介                                               \\\\
  }
 \\end{center}
\\end{titlepage}

\\pagenumbering{roman}
\\tableofcontents
\\listoffigures
\\listoftables

% 序論
\\include{src/introduction}
% 手法
\\include{src/method}
% 実験
\\include{src/experiment}
% まとめ
\\include{src/conclusion}
% 謝辞
\\include{src/acknowledgment}

% 参考文献
\\bibliographystyle{tieice}
\\bibliography{syuron}

% 付録
\\include{src/appendix}
\\include{flow}

\\end{document}
\\typeout{=== Region from (point) to (mark) ===}

\\end{document}
" 0 2475 (fontified nil)) . -1) (t 20737 . 24645) (1 . 2476) (#("%#BIBTEX jbibtex -kanji=euc syuron
% -*- coding: euc-jp -*-
% vim:set fe=euc-jp
% 半角カナ、規格外文字を=に変換 (text-adjust-codecheck-buffer)
% 全角英数文字を半角に置換 (progn (text-adjust-hankaku-buffer) t)
% 句読点を「、。」に置換 (text-adjust-kutouten-buffer)
% 全角文字と半角文字の間に空白を入れる (text-adjust-space-buffer)
% 全て実行 (text-adjust-buffer)
%\\documentclass[a4j,12pt]{jbook}
\\documentclass[a4j,10pt]{jarticle}
%\\documentclass[mentuke,letter]{ieej}
\\usepackage[dvips]{graphics}
\\usepackage[dvipdfmx]{graphicx}
\\usepackage{graphicx}
\\pagestyle{headings}
\\usepackage{enumerate}
\\usepackage{algorithm}
\\usepackage{algorithmic}
%\\newcommand{\\AND}{\\algorithmicand{}}
%\\renewcommand{\\baselinestretch}{1.0}
\\renewcommand{\\baselinestretch}{1.0}
% \\setlength{\\topmargin}{20mm}
% \\addtolength{\\topmargin}{-2in}
% \\setlength{\\textwidth}{193mm}
% \\setlength{\\textheight}{278mm}
% \\setlength{\\oddsidemargin}{10mm}
% \\addtolength{\\oddsidemargin}{-1in}
% \\setlength{\\evensidemargin}{10mm}
% \\addtolength{\\evensidemargin}{-1in}
% \\setlength{\\abovecaptionskip}{0mm}
% \\setlength{\\belowcaptionskip}{0mm}
% \\setlength{\\textheight}{240mm}
%\\renewcommand{\\bibname}{参考文献}
%%% Gceneterの定義
\\newcommand{\\Gcenter}[2]{
  \\dimen0=\\ht\\strutbox%	%					%1行の高さに
  \\advance\\dimen0\\dp\\strutbox% %				%一行の深さを加えて
  \\multiply\\dimen0 by#1% 						%行数で乗じて
  \\divide\\dimen0 by2% %						%等分し
  \\advance\\dimen0 by -.5\\normalbaselineskip% 	%半行だけ上げる.
  \\raisebox{-\\dimen0}[0pt][0pt]{#2}} %		%求めた値の分だけ字下げする
  % 定義
\\newcommand{\\argmax}{\\mathop{\\rm arg~max}\\limits}
\\newcommand{\\argmin}{\\mathop{\\rm arg~min}\\limits}

\\begin{document}

\\begin{document}
\\begin{titlepage}
 \\begin{center}
  \\vspace*{80pt}
  {\\huge
  修士論文                                             \\\\
  オプティカルフローを用いた発話態度の認識に関する研究 \\\\
  \\vspace*{160pt}
  指導教員  木村春彦 教授                              \\\\
  \\vspace*{80pt}
  平成 25 年 1 月 xx 日 提出                          \\\\
  金沢大学 大学院 自然科学研究科                       \\\\
  博士前期課程 電子情報工学専攻                        \\\\
  人工知能研究室                                       \\\\
  脊渓介                                               \\\\
  }
 \\end{center}
\\end{titlepage}

\\pagenumbering{roman}
\\tableofcontents
\\listoffigures
\\listoftables

% 序論
\\include{src/introduction}
% 手法
\\include{src/method}
% 実験
\\include{src/experiment}
% まとめ
\\include{src/conclusion}
% 謝辞
\\include{src/acknowledgment}

% 参考文献
\\bibliographystyle{tieice}
\\bibliography{syuron}

% 付録
\\include{src/appendix}
\\include{flow}

\\end{document}
\\typeout{=== Region from (point) to (mark) ===}

\\end{document}
" 0 2475 (fontified nil)) . -1) (t 20737 . 24637) (1 . 2476) (t 65535 . 65535)))
